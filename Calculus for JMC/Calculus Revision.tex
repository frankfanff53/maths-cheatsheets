\documentclass[12pt]{report}
\usepackage{inputenc}
\usepackage{graphicx}
\graphicspath{{./images/}}
\usepackage{geometry}
\usepackage{amsmath}
\usepackage{amssymb}

\title{MATH40011 Calculus for JMC 
 \\ Revision Note}
\author{Feifan Fan}


\begin{document}
\maketitle
\tableofcontents

\chapter{Differentiation}

\section{Basic Differentiation}

\subsection{Parametric Graph Sketching}

\emph{We are going to sketch a parametric curve $x = f(t)$, $y = g(t)$.}
\newline

\noindent Some basic rules:
\begin{itemize}
    \item Try to find the function $y = h(x)$ between $y$ and $x$
    
    and try to determine the symmetry of the curve: 
    
    i.e. determine whether $h(x) = h(-x)$ 
    
    \emph{(even function, curve is symmetrical about y axis)}

    or $h(x) + h(-x) = 0$ 
    
    \emph{(odd function, curve is symmetrical about the origin)}
    
    or if we have $y^{2} = t(x) \implies y = \pm \sqrt{t(x)}$

    \emph{(not a function, but curve is symmetrical about the x axis}

    \emph{since a value of x points to two values with opposite signs)}
    \item Find the zero points of the curve:
    
    Calculate the value of $t$ such that $y = g(t) = 0$, 
    and calculate the correspnding value of x 
    to get the zero point $(x, 0)$
\end{itemize}
Some special rules for the parametric curves:
\begin{itemize}
    \item Determine when the tangent line to the curve is \emph{horizontal} and \emph{vertical}:
    \newpage
    \emph{Horizontal:} find the value of t such that
    $$
    \frac{dy}{dx} = \frac{\frac{dy}{dt}}{\frac{dx}{dt}} = 0
    $$
     \indent Hence, the tangent is horizontal if
     $$
    \frac{dy}{dt} =  g'(t) = 0 \text{ and }\frac{dx}{dt} = f'(t) \neq 0
     $$
     \emph{Vertical:} find the value of t such that
     $$
     \frac{dy}{dx} = \frac{\frac{dy}{dt}}{\frac{dx}{dt}} = \pm \infty
     $$
     \indent Hence, the tangent is horizontal if
     $$
    \frac{dy}{dt} =  g'(t) \neq 0\text{ and } \frac{dx}{dt} = f'(t) = 0
     $$
    \item Determine at which points is the tangent line to the curve parallel to $y = x$:
    \newline Since for the line $y = x$, $\frac{dy}{dx} = 1$,

    Therefore, if the tangent line is parallel to y = x, 
    $$
    \frac{dy}{dx} = \frac{\frac{dy}{dt}}{\frac{dx}{dt}} = 1
    $$
    \newline Hence, we have 
    $$
    g'(t) = \frac{dy}{dt} = \frac{dx}{dt} = f'(t)
    \text{ where } g'(t) = f'(t) \neq 0$$
    \item Find the limit of the gradient when t is close to 0:
    \newline  $$
    \lim_{t \to 0^+} \frac{dy}{dx} = \lim_{t \to 0^+} \frac{\frac{dy}{dt}}{\frac{dx}{dt}}
    \text{ and }
    \lim_{t \to 0^+} \frac{dy}{dx} = \lim_{t \to 0^+} \frac{\frac{dy}{dt}}{\frac{dx}{dt}}
    $$
    \item Similarly, find the limit of the gradinet when t tends to $\pm \infty$:
    \newline $$
    \lim_{t \to \pm \infty} \frac{dy}{dx} = \lim_{t \to \pm \infty} \frac{\frac{dy}{dt}}{\frac{dx}{dt}}
    $$
    
    In conclusion, we can combine the elements above and find the "dirction" of $t$ to sketch the parametric curve. 
\end{itemize}
 
\newpage
\emph{Example: Sketch the curve}
$$
\left\{\begin{array}{ll}
x = t^2 
\\ y = t^3 - t
\end{array}\right.
$$
\indent in the cartesian plane.

\begin{enumerate}
    \item Firstly we know that $y = t^3 - t = t(t^2 - 1) = t(x - 1)$.
    \newline And we squared y to get:
    $$
    y^2 = t^2(x - 1)^2 = x(x - 1)^2
    $$
    Therefore, 
    $$
    y = \pm \sqrt{x(x - 1)^2}
    $$
    Clearly this curve is symmetrical about x axis. (symmetry $\checkmark$)
    \item 
\end{enumerate}
    


\section{Differentiability \& Continuity of a function}


\end{document} 